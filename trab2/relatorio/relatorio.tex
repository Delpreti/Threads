\documentclass[12pt]{article}

\usepackage[brazilian]{babel}
\usepackage{indentfirst}
% \usepackage{amsmath}
\usepackage{minted}

\author{Gabriel da Fonseca Ottoboni Pinho - DRE 119043838\\
Rodrigo Delpreti de Siqueira - DRE 119022353}
\title{Trabalho II de Computação Concorrente:}
\date{21/05/2021}

\begin{document}
\maketitle
\newpage

\section{Leitores e escritores}
\subsection{O problema}
O modelo de leitores e escritores
tem como objetivo coordenar o acesso
a um certo recurso entre várias threads.
No caso desse trabalho,
o recurso é o array
que contém as medições dos sensores.

A necessidade de coordenar
os acessos a esse array existe
por conta de condições de corrida
que podem ocorrer quando
uma thread tenta ler/escrever
enquanto outra thread já estava escrevendo no array.
Dessa forma, temos os seguintes requisitos:
\begin{itemize}
	\item Se alguma thread estiver escrevedo,
		nenhuma outra thread pode ler ou escrever.
	\item Se uma ou mais threads estiverem lendo,
		nenhuma outra thread pode escrever.
\end{itemize}

Sabendo disso,
é necessário um sistema
no qual cada thread pede permissão
para acessar o array e,
caso o acesso não seja possível naquele momento,
a thread é bloqueada.

\subsection{A solução}
A solução implementada no trabalho
consiste em um \textit{struct} \texttt{Rw}
que armazena os números de threads
atualmente lendo e escrevendo.
Esses números são atualizados
por meio de funções que as threads chamam
antes e depois de iniciar suas tarefas.

Internamente,
o controle de acesso é feito utilizando os \textit{structs}
\texttt{pthread\_\-mutex\_t} e \texttt{pthread\_cond\_t}.
O primeiro garante exclusão mútua durante o acesso
às variáveis que armazenam os números de leitores e escritores,
enquanto que o segundo é utilizado para controlar
o bloqueio das threads que aguardam permissão para ler/escrever.

\newpage
\begin{minted}{c}
Rw rw;
rw_init(&rw);

rw_get_read(&rw);
/* Ler à vontade */
rw_release_read(&rw);

rw_get_write(&rw);
/* Escrever à vontade */
rw_release_write(&rw);

rw_destroy(&rw);
\end{minted}

Como mostrado no exemplo acima,
a função \texttt{rw\_init}
inicializa os campos do \textit{struct},
enquanto que \texttt{rw\_destroy} faz o oposto.
Antes de ler,
a thread deve chamar \texttt{rw\_get\_read},
que retornará imediatamente
se nenhuma outra estiver escrevendo.
Caso contrário, a thread ficará bloqueada
até que nenhuma outra thread esteja escrevendo.
Ao acabar de ler,
a thread deve chamar \texttt{rw\_release\_read}.
As funções \texttt{rw\_get\_write} e \texttt{rw\_release\_write}
funcionam de forma análoga.

Quando há alguma thread aguardando permissão para escrever,
a prioridade para escrita é garantida de duas formas:
\begin{itemize}
	\item Assim que o último leitor chama \texttt{rw\_release\_read},
		é garantido que uma thread será liberada para escrita.
	\item Assim que um escritor chama \texttt{rw\_release\_write},
		é garantido que uma thread será liberada para escrita.
\end{itemize}

\subsection{Testes}

Os testes realizados sobre a implementação que criamos
para o padrão de leitores/escritores visam atender
os seguintes requisitos:
\begin{enumerate}
\item mais de um leitor pode ler ao mesmo tempo
uma área de dados compartilhada;
\item apenas um escritor pode escrever
de cada vez nessa mesma área;
\item enquanto o escritor está escrevendo,
os leitores não podem ler.
\end{enumerate}
Estes requisitos são avaliados individualmente,
no arquivo separado \textit{rwtest.c}.

\section{Monitoramento de temperatura}

A função \texttt{check\_temperature} verifica se haverá
um caso de sinal vermelho, amarelo ou condição normal,
com base nos requisitos do trabalho. O estado exibido
no console apenas é atualizado quando ocorre uma mudança.

\section{Discussão}

Durante a execução, observamos que o Alerta Vermelho
é raramente acionado.

\section{Referências Bibliográficas}

\end{document}
