\documentclass[12pt]{article}

\usepackage[brazilian]{babel}
\usepackage{indentfirst}
\usepackage{titlesec}
\usepackage{minted}

\titleformat{\section}{\normalfont\bfseries}{Questão \thesection}{1em}{}
\titleformat{\subsection}{\normalfont\bfseries}{\Alph{subsection}}{1em}{}
\titleformat{\subsubsection}{\normalfont\bfseries}{\thesubsubsection}{1em}{}
\titlespacing{\section}{0pt}{15pt}{5pt}
\titlespacing{\subsection}{0pt}{5pt}{5pt}

\title{Lista III de Computação Concorrente:}
\author{Gabriel da Fonseca Ottoboni Pinho - DRE 119043838\\
	Rodrigo Delpreti de Siqueira - DRE 119022353}
\date{01/06/2021}

\begin{document}
\maketitle

\section{}
O carro vindo do Norte poderá iniciar a travessia,
pois a ponte está livre.
	
\newpage
\section{}
O erro ocorre quando o
produtor produz enquanto o
consumidor está executando \texttt{consome\_item}.
Se o item consumido for o último do buffer,
\texttt{n} será incrementado pelo produtor
durante o consumo,
fazendo com que o semáforo \texttt{d}
não seja resetado.
Com isso,
\texttt{d} terá valor 2,
o que permite que um
item que não existe seja
retirado do buffer.
Além disso,
o \texttt{if} do consumidor
utiliza a variável \texttt{n}
sem exclusão mútua,
o que pode causar outros problemas.

A solução é
mover o \texttt{if} para
``dentro'' da mutex (semáforo \texttt{s}),
no início do loop,
de modo que a mutex é
devolvida antes de \texttt{sem\_wait},
mantendo a possibilidade de
execução concorrente.
A função \texttt{prod}
permanece inalterada.
\begin{minted}[tabsize=4]{c}
void *cons(void *args) {
	int item;
	while (1) {
		sem_wait(&s);
		if (n == 0) {
			sem_post(&s);
			sem_wait(&d);
			sem_wait(&s);
		}

		retira_item(&item);
		n--;
		sem_post(&s);

		consome_item(item):
	}
}
\end{minted}
\end{document}
