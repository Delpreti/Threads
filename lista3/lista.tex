\documentclass[12pt]{article}

\usepackage[brazilian]{babel}
\usepackage{indentfirst}
% \usepackage{amsmath}
\usepackage{minted}

\title{Lista III de Computação Concorrente:}
\author{Gabriel da Fonseca Ottoboni Pinho - DRE 119043838\\
	Rodrigo Delpreti de Siqueira - DRE 119022353}
\date{01/06/2021}

\begin{document}
	\maketitle
	\newpage
	
	\section{Questão 1}
	O carro vindo do Norte poderá iniciar a travessia,
	pois a ponte está livre.
	
	\newpage
	\begin{minted}{c}
		Rw rw;
		rw_init(&rw);
		
		rw_get_read(&rw);
		/* Ler à vontade */
		rw_release_read(&rw);
		
		rw_get_write(&rw);
		/* Escrever à vontade */
		rw_release_write(&rw);
		
		rw_destroy(&rw);
	\end{minted}
	
	Como mostrado no exemplo acima,
	a função \texttt{rw\_init}
	inicializa os campos do \textit{struct},
	enquanto que \texttt{rw\_destroy} faz o oposto.
	Antes de ler,
	a thread deve chamar \texttt{rw\_get\_read},
	que retornará imediatamente
	se nenhuma outra estiver escrevendo.
	Caso contrário, a thread ficará bloqueada
	até que nenhuma outra thread esteja escrevendo.
	Ao acabar de ler,
	a thread deve chamar \texttt{rw\_release\_read}.
	As funções \texttt{rw\_get\_write} e \texttt{rw\_release\_write}
	funcionam de forma análoga.
	
	Quando há alguma thread aguardando permissão para escrever,
	a prioridade para escrita é garantida de três formas:
	\begin{itemize}
		\item Assim que o último leitor chama \texttt{rw\_release\_read},
		é garantido que uma thread será liberada para escrita.
		\item Assim que um escritor chama \texttt{rw\_release\_write},
		é garantido que uma thread será liberada para escrita.
		\item Se há um escritor bloqueado, todos os novos leitores
		também serão bloqueados.
	\end{itemize}
	
	\newpage
	\subsection{Testes}
	
	Os testes realizados sobre a implementação que criamos
	para o padrão de leitores/escritores visam atender
	os seguintes requisitos:
	\begin{enumerate}
		\item mais de um leitor pode ler ao mesmo tempo
		uma área de dados compartilhada;
		\item apenas um escritor pode escrever
		de cada vez nessa mesma área;
		\item enquanto o escritor está escrevendo,
		os leitores não podem ler.
		\item Quando uma escritora está escrevendo e
		há ambas threads leitoras e escritoras bloqueadas,
		as escritoras devem ter prioridade.
		\item Quando há leitoras lendo e uma escritora
		é bloqueada, futuras threads leitoras
		também precisam ser bloqueadas.
	\end{enumerate}
	Estes requisitos são avaliados individualmente,
	no arquivo separado \textit{rwtest.c}.
	
	\section{Monitoramento de temperatura}
	
	A solução utiliza uma struct para guardar os valores
	respectivos de temperatura para cada thread. Estas
	recebem o próprio id e um ponteiro que a solução
	de leitores e escritores exige. Esse ponteiro é
	utilizado apenas nos testes, não faz parte da lógica
	do programa principal.
	
	O programa irá gerar valores aleatórios de temperatura,
	dentro do intervalo de 25 a 40 graus Celsius,
	utilizando a função \texttt{get\_temperature\_rand}.
	Os valores são gerados com uma casa decimal, conforme
	orientação do trabalho. Intuitivamente, entendemos que
	esta função não condiz com uma situação real,
	onde a variação da temperatura ocorreria de forma contínua.
	Todavia, mantivemos este método por ser de fácil imlementação.
	
	A função \texttt{check\_array} é executada pela thread leitora
	(atuador). Ela verifica se haverá
	um caso de sinal vermelho, amarelo ou condição normal,
	com base nos requisitos do trabalho. Além disso, ela
	exibe a média 
	
	A função \texttt{add\_temp} é executada pela thread escritora
	(sensor). Ela insere o valor medido no espaço
	de memória compartilhado.
	
	É permitido ao usuário interagir com a aplicação
	definindo o número de threads. Após mandar rodar,
	a aplicação executará por tempo indefinido, até ser encerrada
	por Ctrl+C.
	
	\newpage
	\section{Discussão}
	
	Durante a execução, observamos que o Alerta Vermelho
	é raramente acionado. Um dos motivos identificados
	vem do que discutimos anteriormente: a função que
	gera os valores de temperatura não trata o intervalo
	de valores como contínuo. Isso torna muito mais improvável
	a ocorrência de 5 leituras seguidas com temperatura superior
	a 35 graus.
	
	Além disso, o alerta amarelo também demora algumas
	execuções para aparecer, pois no início não há escritas
	o suficiente para que tais condições sejam atendidas.
	
\end{document}
