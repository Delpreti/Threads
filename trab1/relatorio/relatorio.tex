\documentclass[12pt]{article}

\usepackage[brazilian]{babel}
\usepackage{indentfirst}
\usepackage{amsmath}
\usepackage{xfrac}
\usepackage{lmodern}
\usepackage[backend=biber,block=ragged]{biblatex}
\usepackage{csquotes}
\usepackage{pgfplots}
\usepackage{multirow}

\pgfplotsset{width=7cm,compat=1.17}
\addbibresource{referencias.bib}
\DeclareMathOperator{\sha}{SHA256}

\author{Gabriel da Fonseca Ottoboni Pinho - DRE 119043838\\
Rodrigo Delpreti de Siqueira - DRE 119022353}
\title{Trabalho I de Computação Concorrente: \\Simulação de Mineração de Bitcoin}
\date{25/04/2021}

\begin{document}
\maketitle
\newpage
\tableofcontents
\newpage

\section{Descrição do problema}
\subsection{Sobre a Bitcoin}
Bitcoin é uma criptomoeda criada por Satoshi Nakamoto em 2009,
tendo como objetivo ser uma moeda digital descentralizada.
Uma parte essencial da Bitcoin é uma tecnologia chamada \textit{Blockchain},
que consiste em armazenar todas as transações que já ocorreram
em blocos interligados.
Cada transação é criptograficamente assinada por seu emissor
e então enviada ao resto da rede,
que verifica a assinatura e a existência dos fundos.

Um ponto importante é que cada transação é verificada por todos,
de modo que nenhum integrante precisa confiar em nenhum outro,
cada um sendo capaz de verificar a verdade independentemente.
Os integrantes da rede que coletam as transações
e as colocam em blocos são os mineradores
e o sistema de consenso que determina qual minerador
terá o direito de criar o bloco se chama \textit{Proof of Work} (PoW).

A ideia da PoW é que o minerador que conseguir
a resposta de um desafio primeiro
tem o direito de criar o bloco.
Esse desafio consiste em achar um número \textit{nonce},
tal que o \textit{hash} SHA-256 do \textit{header} do bloco
(que contém a \textit{nonce})
seja menor que um número chamado de dificuldade.
Pelo fato do SHA-256 ser uma função \textit{hash} criptográfica,
a única forma de achar uma \textit{nonce} correta é
testando valores.
Em outras palavas, não há um método fácil de achar
um $x$ tal que $\sha(x) = y$.
Por outro lado, tendo um $x$,
é fácil verificar se $\sha(x) = y$.
Essa propriedade é importante,
pois, dessa forma, todos os integrantes da rede podem
verificar facilmente se a \textit{nonce} encontrada
é uma solução válida de fato.

\subsection{Sobre o SHA-256}
O SHA-256 é uma função \textit{hash} criptográfica
que gera uma sequência de 256 bits pseudo-aleatória
para uma entrada qualquer.
O cálculo dessa função é computacionalmente custoso,
e é o \textit{work} na PoW da Bitcoin.
O cálculo da função consiste em 3 passos principais:
\begin{enumerate}
	\item Pré-processamento
	\item Criação do array de mensagens
	\item Compressão
\end{enumerate}

Durante o pré-processamento,
a entrada é dividida em
pedaços chamados \textit{chunks} de 512 bits cada.
Depois disso, para cada \textit{chunk},
um array de 64 elementos de 4 bytes é
preenchido com base no conteúdo do \textit{chunk} atual.
Por fim, 8 variáveis são inicializadas com valores pré-determinados
que serão modificados em cada \textit{round} do loop de compressão.
O valor final do \textit{hash} será a concatenação das 8 variáveis.

\section{Projeto da solução concorrente}
O objetivo do nosso trabalho é
implementar o SHA-256 e
usá-lo para simular a mineração da Bitcoin.
Em outras palavras,
temos que receber como entrada
um número $d$ representando a dificuldade e
um número $n$ de threads a serem utilizadas.
A saída será uma \textit{nonce} que
resolveria um bloco com essa dificuldade e
o tempo levado para calculá-la.
A dificuldade será o número de vezes seguidas 
que o caractere \texttt{\textquotesingle{}b\textquotesingle{}} aparece no
início da representação hexadecimal do \textit{hash} calculado.
Como foi explicado na sessão anterior,
não é exatamente assim que a dificuldade é definida
na implementação da Bitcoin,
mas foi decidido simplificar essa etapa.

A simulação funcionará da seguinte forma:
\begin{enumerate}
	\item 76 bytes aleatórios são gerados.
		Esses bytes representarão todos os campos
		do \textit{header} do bloco,
		exceto a \textit{nonce}.
	\item Cada thread calculará o \textit{hash} da concatenação
		desses mesmos 76 bytes com mais 4 bytes, a \textit{nonce}.
		Temos assim um total de 80 bytes,
		que é o tamanho real do \textit{header} de
		um bloco de Bitcoin.
	\item Se a representação hexadecimal de $\sha(\text{\textit{header}})$
		começar com pelo menos $d$ caracteres \texttt{\textquotesingle{}b\textquotesingle{}} seguidos,
		conseguimos achar uma \textit{nonce} correta; fim.
		Senão, repetimos as etapas 2 e 3
		com uma \textit{nonce} diferente.
\end{enumerate}

É importante notar que cada tentativa
é completamente independente da outra, ou seja,
podemos facilmente acrescentar mais threads,
cada uma testando diferentes valores para a \textit{nonce}
até que a resposta correta seja encontrada.
A estratégia utilizada foi
testar valores consecutivos para a \textit{nonce}
de modo que esses números são distribuídos
entre as $n$ threads de $n$ em $n$.
Os valores necessários para o cálculo do \textit{hash}
são passados por argumentos para as threads e,
quando alguma das threads achar a resposta,
uma variável global \textit{flag} é setada
e as outras threads terminam.
A thread que encontrou a resposta
retorna seu valor para a thread principal,
que imprime os resultados na tela.

\subsection{Do repositório}
O repositório contém um grupo de arquivos,
cuja função está brevemente descrita aqui para fins de organização.
O algoritmo que calcula propriamente o hash sha-256 está contido no arquivo \textmd{sha256.c}, como uma função.
Esta função recebe um ponteiro que aponte para o header do bloco, o tamanho do header e um ponteiro usado para retornar o resultado como string.
Ela deve ser acessada pelas outras partes do programa utilizando o arquivo \textmd{sha256.h} disponibilizado.

O programa propriamente dito,
para cumprir os objetivos acima citados,
está no arquivo \textmd{threadmine.c}.
Ele recebe como parâmetros a dificuldade do algoritmo
e o número de threads que serão utilizadas.
Assim, chamamos esse programa múltiplas vezes
em \textsl{script.sh} para gerarmos os resultados
necessários para análise posterior.
Os valores utilizados na análise estão todos contidos
no arquivo de texto \textsl{saida.txt}.

Por fim, os casos de teste estão separados no arquivo \textsl{testsha.c}. Há uma \texttt{makefile} disponibilizada, de modo que para compilar o programa basta clonar o repositório e utilizar o comando \texttt{make}; para compilar e executar os testes, \texttt{make test}. Para executar o programa em si, basta chamar \texttt{./threadmine <dificuldade> <numero de threads>}

\section{Casos de teste}
A principal função do programa que
precisa ser testada é \texttt{sha256},
que recebe uma sequência de bytes e
retorna uma string com o valor hexadecimal
do \textit{hash} SHA-256 da entrada.
Cada teste chama essa função com
valores de entrada pré-determinados e
compara o \textit{hash} retornado
com um valor correto pré-calculado.

Além disso, foi utilizado um \textit{shell script}
para executar o programa com
diversos valores de $n$ e $d$ múltiplas vezes.
Os resultados dessas execuções
foram utilizados na próxima seção.

\newpage
\section{Avaliação de desempenho}
Realizamos 10 execuções para cada configuração de dificuldade e thread.
Devido à natureza do problema,
iremos utilizar para fins comparativos
a média de tempo dessas 10 execuções.
Essa quantidade pode não gerar ainda
grande confiabilidade estatística do resultado,
porém acreditamos ser suficiente
para a devida análise e escopo deste trabalho.
Seria ingênuo testar execuções individuais,
pois a aleatoriedade dos valores pode levar
execuções individuais a apresentarem tempos muito díspares.

A tabela abaixo apresenta o tempo médio das 10 execuções, para 7 dificuldades em ordem crescente, com 1, 2, 4 e 8 threads. A máquina utilizada roda Windows 10, com um processador AMD Ryzen 7 3700x, que possui 8 núcleos físicos e 16 núcleos lógicos.

\begin{center}
\resizebox{\textwidth}{!}{%
	\begin{tabular}{|c|c|c|c|c|c|c|c|}
		\hline
		\multirow{2}{*}{Threads} & \multicolumn{7}{c|}{Dificuldade} \\
		\cline{2-8}
		 & 1 & 2 & 3 & 4 & 5 & 6 & 7 \\
		\hline
		1 & 0,000206 & 0,000831 & 0,010532 & 0,145358 & 2,354084 & 20,96 & 400,6 \\
		\hline
		2 & 0,000249 & 0,000450 & 0,003139 & 0,036904 & 1,235175 & 13,27 & 284,26 \\
		\hline
		4 & 0,000372 & 0,000455 & 0,002497 & 0,023467 & 0,508354 & 7,34 & 76,56 \\
		\hline
		8 & 0,000755 & 0,000744 & 0,001785 & 0,012743 & 0,346573 & 7,24 & 93,16 \\
		\hline
	\end{tabular}
}
\end{center}

\begin{center}
\begin{tikzpicture}
\begin{semilogyaxis}[
	xlabel=Threads,
	ylabel=Tempo (log s),
	xtick={1,2,4,8},
	legend entries={$d=1$,$d=2$,$d=3$,$d=4$,$d=5$,$d=6$,$d=7$},
	legend style={at={(1.4,1)}}
]
\addplot coordinates {(1,0.000206) (2,0.000249) (4,0.000372) (8,0.000755)};
\addplot coordinates {(1,0.000831) (2,0.000450) (4,0.000455) (8,0.000744)};
\addplot coordinates {(1,0.010532) (2,0.003139) (4,0.002497) (8,0.001785)};
\addplot coordinates {(1,0.145358) (2,0.036904) (4,0.023467) (8,0.012743)};
\addplot coordinates {(1,2.354084) (2,1.235175) (4,0.508354) (8,0.346573)};
\addplot coordinates {(1,20.96) (2,13.27) (4,7.34) (8,7.24)};
\addplot coordinates {(1,400.6) (2,284.26) (4,76.56) (8,93.16)};
\end{semilogyaxis}
\end{tikzpicture}
\end{center}

\begin{center}
\begin{tikzpicture}
\begin{semilogyaxis}[
	xlabel=Dificuldade,
	ylabel=Tempo (log s),
	xtick={1,2,3,4,5,6,7},
	legend entries={$n=1$,$n=2$,$n=4$,$n=8$},
	legend style={at={(1.4,1)}}
]
\addplot coordinates {(1,0.000206) (2,0.000831) (3,0.010532) (4,0.145358) (5,2.354084) (6,20.96) (7,400.6)};
\addplot coordinates {(1,0.000249) (2,0.000450) (3,0.003139) (4,0.036904) (5,1.235175) (6,13.27) (7,284.26)};
\addplot coordinates {(1,0.000372) (2,0.000455) (3,0.002497) (4,0.023467) (5,0.508354) (6,7.34) (7,76.56)};
\addplot coordinates {(1,0.000755) (2,0.000744) (3,0.001785) (4,0.012743) (5,0.346573) (6,7.24) (7,93.16)};
\end{semilogyaxis}
\end{tikzpicture}
\end{center}

Podemos observar diretamente que
a utilização de threads não foi vantajosa
para as dificuldades 1 e 2.
O tempo gasto com overhead de criação das threads
foi superior ao ganho de velocidade.

Outro resultado muito interessante do experimento
é que o tempo se torna mais consistente
com um número maior de threads.
Quanto mais valores são testados para um mesmo intervalo de tempo,
maiores as chances de obter tempos similares entre as execuções.
Olharemos para a variança dos resultados em dificuldade 5 como exemplo:

\hspace{9px}

\begin{tabular}{|c|c|c|c|c|}
	\hline
	& \multicolumn{4}{c|}{Threads}\\
	\hline
	& 1	& 2	& 4	& 8\\
	\hline
	1 & 1,81866	& 2,25835 & 0,12451	& 0,84497 \\
	\hline
	2 & 0,7952 & 2,07361 & 0,19844 & 0,14135 \\
	\hline
	3 & 1,36829 & 0,12525 & 0,24746 & 0,34731 \\
	\hline
	4 & 0,34141 & 0,80404 & 1,90943 & 0,36996 \\
	\hline
	5 & 0,64075 & 4,27544 & 0,37336 & 0,03728 \\
	\hline
	6 & 0,2022 & 0,17607 & 0,88296 & 0,10144 \\
	\hline
	7 & 6,07897 & 0,75382 & 0,50713 & 0,56717 \\
	\hline
	8 & 4,3207 & 0,63343 & 0,31499 & 0,4524 \\
	\hline
	9 & 5,05224 & 1,1545 & 0,39751 & 0,10372 \\
	\hline
	10 & 2,92242 & 0,09724 & 0,12775 & 0,50013 \\ 
	\hline
	Média & 2,354084 & 1,235175 & 0,508354 & 0,346573 \\
	\hline
	Variança & 4,065096858 & 1,54277901 & 0,262755251 & 0,058923561 \\
	\hline
\end{tabular}

\hspace{20px}

Na tabela acima, vemos que a redução na variança
é mais acentuada que a redução do tempo médio de execução.

\section{Discussão}
\subsection{Análise dos resultados}
Apesar do problema possuir diversas variáveis de cunho aleatório,
podemos estimar razoavelmente o tempo de execução
a partir da probabilidade de se encontrar um \textit{hash} com determinada dificuldade.
Conforme especificado no código,
para que a execução termine adequadamente,
os $d$ primeiros caracteres
do \textit{hash} gerado como resultado
da \texttt{sha256} devem ser iguais a \texttt{\textquotesingle{}b\textquotesingle{}},
onde $d$ é o valor da dificuldade
passado como parâmetro na chamada de \texttt{threadmine}.
Existem 16 caracteres possíveis para cada posição,
pois o algoritmo retorna como string os valores em hexadecimal.
Supondo que a função de \textit{hash}
apresente uma distribuição uniforme de valores,
a probabilidade de acerto será de $\sfrac{1}{16^{n}}$.

Este resultado nos indica que o tempo médio
deverá aumentar exponencialmente com a dificuldade.
Essa estimativa foi verificada na prática,
como podemos observar no gráfico Tempo $\times$ Dificuldade
(note que o tempo está em escala logarítmica).
Por tal razão, os testes foram realizados apenas até a dificuldade 7:
para dificulade 8, o tempo estimado de execução
para uma única thread seria superior a uma hora.

\subsection{Intel® SHA Extensions}
Apesar de não terem sido utilizadas nesse trabalho,
processadores modernos possuem 3 instruções
dedicadas ao cálculo do SHA-256, que são:
\begin{itemize}
	\item \texttt{SHA256MSG1}
	\item \texttt{SHA256MSG2}
	\item \texttt{SHA256RNDS2}
\end{itemize}

As duas primeiras aceleram o cálculo
do array de mensagens,
enquanto a última calcula
duas iterações do loop de compressão.
Essas instruções utilizam registradores de 128 bits,
que possibilitam um grande número de entradas para cada instrução.
O fato de o cálculo dessa função fazer parte dos processadores modernos
mostra o quão relevante o SHA-256 realmente é,
tendo aplicações muito além da Bitcoin.
O uso dessas intruções seria uma maneira de
melhorar a performance da simulação,
acelerando o cálculo do \textit{hash} dramaticamente.

\nocite{*}
\printbibliography

\end{document}
