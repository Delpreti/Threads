\documentclass[12pt]{article}

\usepackage[brazilian]{babel}
% \usepackage[margin=1in]{geometry}
\usepackage{minted}
\usepackage{indentfirst}
\usepackage{amsmath}

\DeclareMathOperator{\sha}{SHA256}

\author{Gabriel da Fonseca Ottoboni Pinho - DRE 119043838\\
Rodrigo Delpreti de Siqueira - DRE 119022353}
\title{Trabalho I de Computação Concorrente: \\Simulação de Mineração de Bitcoin}
\date{25/04/2021}

\begin{document}
\maketitle
\newpage

\section{Descrição do trabalho}
\subsection{Sobre a Bitcoin}
Bitcoin é uma criptomoeda criada por Satoshi Nakamoto em 2009,
tendo como objetivo ser uma moeda digital descentralizada.
Uma parte essencial da Bitcoin é uma tecnologia chamada \textit{Blockchain},
que consiste em armazenar todas as transações que já ocorreram
em blocos interligados.
Cada transação é criptograficamente assinada por seu emissor
e então enviada ao resto da rede,
que verifica a assinatura e a existência dos fundos.

Um ponto importante é que cada transação é verificada por todos,
de modo que nenhum integrante precisa confiar em nenhum outro,
sendo capaz de verificar a verdade independentemente.
Os integrantes da rede que coletam as transações
e as colocam em blocos são os mineradores
e o sistema de consenso que determina qual minerador
terá o direito de criar o bloco se chama \textit{Proof of Work} (PoW).

A ideia da PoW é que o minerador que conseguir
a resposta de um desafio primeiro
tem o direito de criar o bloco.
Esse desafio consiste em achar um número \textit{nonce},
tal que quando o \textit{hash} SHA-256 do \textit{header} do bloco
(que contém a \textit{nonce})
seja menor que um certo número.
Pelo fato do SHA-256 ser uma função \textit{hash} criptográfica,
a única forma de achar a \textit{nonce} correta é chutando.
Em outras palavas, não é possível achar um
$x$ tal que $\sha(x) = y$.
Por outro lado, tendo um $x$, é fácil verificar se $\sha(x) = y$.
Essa propriedade é importante,
pois, dessa forma, todos os integrantes da rede podem
verificar facilmente se a \textit{nonce} encontrada
é uma solução válida de fato.

\subsection{Sobre o SHA-256}
O SHA-256 é uma função \textit{hash} criptográfica,
que gera uma sequência de 256 bits pseudo-aleatória
para uma entrada qualquer.
O cálculo dessa função é computacionalmente custoso,
e é \textit{work} na PoW da Bitcoin.
O cálculo da função consiste em 3 passos:
\begin{enumerate}
	\item Pré-processamento
	\item Criação do array de mensagens
	\item Compressão
\end{enumerate}

Durante o pré-processamento,
a entrada é será divida em pedaços chamado \textit{chunks}
de 512 bits cada.
Depois disso, para cada \textit{chunk},
um array de 64 elementos de 4 bytes é
preenchido com base no conteúdo do \textit{chunk} atual.
Por fim, 8 variáveis são inicializadas com valores pré-determinados
que serão modificados em cada \textit{round} do loop de compressão.
O valor final do \textit{hash} será a concatenação das 8 variáveis.

\section{A solução}
O objetivo do nosso trabalho é
implementar o SHA-256 e
usá-lo para simular a mineração da Bitcoin.
Em outras palavras,
vamos gerar 76 bytes aleatórios,
que simularão o \textit{header} de um bloco da Bitcoin.
Em seguida, acrescentaremos mais 4 bytes,
que representarão a \textit{nonce},
totalizando 80 bytes.
Depois, calcularemos o \textit{hash} SHA-256 desses 80 bytes:
se o resultado for menor que um certo número $n$,
a \textit{nonce} é uma solução válida.
Senão, alteramos a \textit{nonce} e tentamos novamente

É importante notar que cada tentativa
é completamente independente da outra, ou seja,
podemos facilmente acrescentar mais threads,
com cada um testando diferentes valores para a \textit{nonce}
até que a resposta certa seja encontrada.

\end{document}
